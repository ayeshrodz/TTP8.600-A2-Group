\section{Potential Users or User Groups of the Solution}

The solution identifies seven potential user groups, describes as follows. 


\begin{itemize}
    \item \textbf{Professional Parents:} Parents with professionals are searching for reliable and efficient childcare centers for their kids. Those who balance both their career and family responsibilities search for a value quick, comfort, real-time updates, and communication about child’s daily activities, frequently depending on technology to organize and optimize these exchanges.
    \item \textbf{Childcare Provider:} Childcare provider handles the process of childcare facilities such as staff management, communication with parent group and observation to regulatory standards. This group searches for efficient tools to manage administrative duties, enhance parent engagement, concerns on safety and privacy, educational productivity, staff coordination and smooth processing of the center.
    \item \textbf{Caretaker Job Seekers:} Those who are searching for job opportunities in childcare centers desire a platform that make things easier for process of search jobs including highlights matching with their job category, qualifications, preferences, simplifies the process of applying for jobs and communicating with potential employers.
    \item \textbf{New Migrants Parents:} Specifically in this group who are with young children, need assistance for navigating the childcare in a new country. They require accessible data on available childcare options, educational outlines, cultural adoption to make informed decisions for their children’s care and progress. 
    \item \textbf{Ministry of education:} The Ministry of Education frames policies, educational standards, affords funding and support to childcare centers. This group of user category is interested in tools which help to monitor, enhance educational outcomes and ensure equitable access to quality childcare centers from different regions.
    \item \textbf{Development Team:} Development team who handles the technical architecture behind the mobile application, responsible for translating user requirements into an efficient and user-friendly solution. They need clear requirements and feedback from all the categories of users to develop the app that challenges come up and improve the childcare experience for parent group, providers and staff of educators.
    \item \textbf{Kindergarten associations:} This user group represents a collective of kindergartens, supporting childcare staff, facilitating professional growth, and supporting early childhood education. They are looking for solutions that promote the best applies, support the professional development of educators, and improve operational productivity.
\end{itemize}

In the next section, a detailed exploration was conducted through contextual inquiries and interviews with 3 individuals representing randomly selected 3 users from 3 different user groups are included as \emph{Appendix A}.

