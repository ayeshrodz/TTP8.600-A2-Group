
\section{Background Study and Analysis}
As in many urban areas in world, the childcare industry in New Zealand consists of a variety of offerings, from private day care centers to community-based services. Around 96.8\% of children in New Zealand attend childcare centers while every center is guided by the curriculum framework of “Te Whariki” \cite{ministryofeducation_2023}. However, parents always face difficulties in finding \& selecting the right childcare service due to a lack of centralized information. They must navigate through various resources to get details about locations, availability, staff, facilities, and rates, which can be time-consuming and stressful. Searching for data on different platforms can waste considerable time and inefficiencies. Each resource can have its own interface, search parameters and methodologies for performing data, which increases the duration required to collect all essential information. 

The flow of collecting information from dissimilar sources can be mentally demanding. It includes safeguarding accuracy, categorizing through many data, and comparing the best option. The parental side observes their decision-making flow regarding childcare as inactive and often depends on recommendations from others rather than actively discovering numerous options. This suggests a tendency towards accepting default options rather than actively engaging in a comparative analysis of available alternatives \cite{Barraclough_Smith_1996}. An Early Childcare center is considered a child’s first environment outside of their home where they engage in both socialization and structural education \cite{Fenech_Sumsion_2007}. Therefore, modern parents are more concerned about the quality of education provided by teachers in care centers.