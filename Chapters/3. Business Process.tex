\section{Business Process}
The ministry of education provides a certain amount of funding for three to five years old up to a maximum of 6 hours per child per day and 20 hours per week as subsidy. \cite{ministryofeducation_2023} The number of children a childcare center can enroll and maintain depends on how big it is and where it is located. Info care is currently the most widely used system in childcare centers. The center’s business process encompasses several key aspects:

\begin{itemize}
    \item \textbf{Enrollment} – Information is received about the center’s programs, philosophy, and enrollment procedures and followed by registration using manual paperwork for billing and tracking attendance.
    \item \textbf{Registration} - Parents interested in enrolling their children to the center fill in the form and schedule a visit to the center and move for the orientation program. 
    \item \textbf{Communication} - Communication in childcare centers is multifaceted and vital for maintaining strong relationships between parents and caregivers. Regular updates and reports are provided to offer detailed insights into a child's behavior, keeping parents informed and engaged in their child's daily experiences. Information about upcoming events and activities are communicated through newsletters or digital platforms, allowing parents to participate and plan accordingly. In times of emergency, clear protocols ensure swift communication with parents, providing reassurance and updates as needed. \cite{mitchell2006teachers}
    \item \textbf{Safety and Security} - Safety and security in childcare centers are upheld through a multi-layered approach. It begins with an inclusive risk valuation of the premises to identify potential threats, followed by the growth and operation of stringent safety policies and procedures. Staff members undergo rigorous training to ensure they are equipped to respond effectively to emergencies and maintain proper supervision of the children. Secure entry and exit procedures are enforced, along with regular drills to prepare for various emergencies. Furthermore, childcare centers prioritize well-being and cleanliness practices to prevent the spread of illness. Transparent communication with parents about safety measures and continuous evaluation and improvement of protocols ensure that the center preserves a secure environment for the well-being of the children under their care.  
    \item \textbf{Emergency Preparedness Plans} - Childcare centers are designed with safety features, including equipment for fire suppression, sensors to detect harmful gases, and first aid kits. child-sized furniture and equipment are also selected and arranged to minimize injury risks.
    \item \textbf{Staff Training} - All staff members receive training in safety procedures, including CPR, first aid, and basic childcare safety protocols. They are also trained to recognize signs of abuse or neglect and understand reporting procedures. \cite{Fenech_Sumsion_2007}
    \item \textbf{Child Supervision} - Childcare centers always maintain strict supervision of children, both indoors and outdoors. Ratios of children to staff members are typically set and monitored to ensure adequate supervision and individual attention. 
    \item \textbf{Health and Hygiene Practices} - Centers enforce strict health and hygiene practices to prevent the spread of illness and maintain a clean environment. This includes regular handwashing for children and staff, proper diapering and toileting procedures, and routine cleaning and sanitizing of toys, surfaces, and common areas. 
    \item \textbf{Background Checks} - As mentioned earlier, thorough background checks are conducted on all staff members to ensure they have no history of criminal activity or child abuse. 
    \item \textbf{Child Pick-Up Policies} - Centers implement strict policies for child pick-up, requiring authorized individuals to present identification and sign children in and out. This helps prevent unauthorized individuals from accessing the facility or removing children without permission.
    \item \textbf{Continuous Monitoring and Improvement} - Childcare centers constantly evaluate and update their safety protocols to ensure compliance with legal standards. This ongoing process of monitoring and enhancement is essential to keep the environment secure for children. 
\end{itemize}

By implementing these safety and security methods, childcare centers strive to create a nurturing and protective environment where children can learn, grow, and thrive.