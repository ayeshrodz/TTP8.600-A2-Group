\section{Analysis and Justification for the proposed solution.}
As per the analysis, the issue of disconnected childcare centers in Auckland underscores the need for a centralized communication platform to streamline the enrollment process to improve coordination between parents and centers. Addressing this problem can enhance accessibility, efficiency, and satisfaction for all stakeholders involved.

\subsection{Analysis of potential solutions.}
The issue at hand revolves around the lack of connectivity and centralized communication among individual childcare centers in Auckland. Considering the facts and it appeared four, but not limited solutions as follows: 

\begin{itemize}
    \item \textbf{National Online Portal} - Develop a government-backed online portal that serves as a centralized platform for all childcare services across New Zealand.\\ \\
    \emph{\textbf{Benefits:}} This portal could provide comprehensive information on available childcare centers, including location, capacity, fees, programs offered, and staff qualifications. It would allow parents to apply to multiple centers through a single application form, track their application status, and receive updates. \\
    \emph{\textbf{Limitations:}} The Early Childhood Council has already raised several concerns to the new government. \cite{daycareshortage} However, building such a centralized national platform could take a considerable time and will not be sufficient to address the current situation at hand as it requires an immediate solution.
\end{itemize}