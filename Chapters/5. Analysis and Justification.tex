\section{Analysis and Justification for the proposed solution.}
As per the analysis, the issue of disconnected childcare centers in New Zealand underscores the need for a centralized communication platform to streamline the enrollment process to improve coordination between parents and centers. Addressing this problem can enhance accessibility, efficiency, and satisfaction for all stakeholders involved. \par

\subsection{Analysis of potential solutions.}
The issue at hand revolves around the lack of connectivity and centralized communication among individual childcare centers in New Zealand. Considering the facts and it appeared four, but not limited solutions as follows: 
\begin{itemize}
    \item \textbf{National Online Portal} - Develop a government-backed online portal that serves as a centralized platform for all childcare services across New Zealand.\\ \\
    \emph{\textbf{Benefits:}} This portal could provide comprehensive information on available childcare centers, including location, capacity, fees, programs offered, and staff qualifications. It would allow parents to apply to multiple centers through a single application form, track their application status, and receive updates.

    \emph{\textbf{Limitations:}} The Early Childhood Council has already raised several concerns to the new government. \cite{daycareshortage} However, building such a centralized national platform could take a considerable time and will not be sufficient to address the current situation at hand as it requires an immediate solution.

    \item \textbf{Web Site} - Creating a website to function as a centralized platform for all childcare services.\\ \\
    \emph{\textbf{Benefits:}} Selecting a web-based application for centralized childcare centers will streamline operations, improve parent communication, ease managing regulatory compliance in childcare industry, and provide a safe environment for children.

    \emph{\textbf{Limitations:}} Accessibility will be an issue due to the requirement of internet for every access. This may have communication interruptions. As far as security is concerned, there is always a risk of data breaches. A web-based method of communication will be costly.

    \item \textbf{Existing Applications} - Select and promote an existing childcare application that has subscriptions in the market to help address this issue.\\ \\
    \emph{\textbf{Benefits:}} Using existing applications for centralized childcare centers helps streamline related operations and will avoid investing for developing such a solution.

    \emph{\textbf{Limitations:}} The government has issued specific guidelines for handling personal information offshore, stressing due diligence regardless of where providers process data. Providers are required to protect personal data, while agencies remain accountable for its privacy. \cite{nzdigitalgovernment} However, the prevalence of applications storing data in various countries poses limitations.

    \item \textbf{Mobile Application} - Create a mobile app for New Zealand market, making it easier for parents to search for childcare services on-the-go.\\ \\
    \emph{\textbf{Benefits:}} Mobile applicants improve efficiency and productivity by streamlining communications and tasks \cite{sarwar2013impact}. Enhance user engagement through personalized, interactive experiences and leverage smartphone features like GPS and cameras for innovative functionalities. Personalization improves customer service and loyalty, while accessibility allows for the convenience of using services anytime, anywhere \cite{shaikh2015mobile}.

    \emph{\textbf{Limitations:}} Data privacy and security concerns due to the sensitivity of children's information.\cite{privacyact2020} The digital divide presents access challenges for some families, potentially exacerbating inequalities \cite{di2020likely}. Over-reliance on technology can also reduce personal interaction between parents and caretakers. The need for substantial investment in training and infrastructure poses a challenge for resource-limited settings.
\end{itemize}


\subsection{Justification for selecting the solution}

In summary, creating a Mobile Application, offers a comprehensive and forward-thinking approach to addressing the childcare needs of parents in New Zealand. By embracing mobile technology, we can deliver a user-centric, accessible, and engaging platform that empowers parents with the tools and information they need to make informed decisions about their child's care. Therefore, out of the above discussed solutions, it appears developing a centralized mobile application is a more viable option. \par